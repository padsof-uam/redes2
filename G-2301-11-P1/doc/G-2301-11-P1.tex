\documentclass[nochap]{apuntes}

\begin{document}

\section{Rendimiento}

Hemos hecho varias pruebas para verificar el rendimiento de nuestro servidor. Para ello hemos usado el script \texttt{crazymonkey.py}, que crea múltiples hilos que se conectan al servidor. 

La medida del tiempo de respuesta se hace con el comando \textit{PING}. El hilo principal del script envía un comando \textit{PING} y mide el tiempo que tarda hasta recibir el correspondiente \textit{PONG}. No han sido necesarios identificadores ya que en ningún momento el script está esperando respuesta a más de un \textit{PING}.

En cuanto a los hilos auxiliares, escogen un comando aleatorio de una lista predefinida:

\begin{verbatim}
	"NICK {nick}",
	"JOIN {channel}",
	"PART {channel}",
	"PRIVMSG {user} :{message}",
	"PRIVMSG {channel} :{message}"
\end{verbatim}

Esto es, comandos de cambio de nombre, entrada y salida de canales y envío de mensajes a usuarios y canales. Tanto los usuarios como los canales se obtienen de listas fijas, globales para todos los hilos, de tal forma que se simula una actividad más o menos real.

\section{Arquitectura}

\subsection{Explicación}

Nuestro servidor funciona con 4 hilos, que a grandes rasgos se distribuyen así:
\begin{itemize}
\item  Un hilo a la escucha de nuevas conexiones. (\emph{listener})
\item Un hilo recibiendo datos de esas conexiones (\emph{receiver}) y encolando los mensajes mediante una cola de sistema con la que se comunica con el principal.
\item El hilo principal, que gestiona las estructuras de datos y realiza los cambios necesarios a partir de los mensajes. Es el encargado de parsear los mensajes también y de encolarlos (con otra cola de sistema) para el hilo encargado de eso.
\item El hilo que envía los mensajes (\emph{sender}). Recibe los mensajes encolados para enviarlos.
\end{itemize}

En la estructura de datos globales de irc (irc_globdata) tenemos 2 diccionarios de usuarios. Unos cuya clave es el descriptor del socket de comunicación y otro cuya clave es el nick, pero ambos tienen como valor el mismo puntero a la misma dirección de memoria. Esta decisión, aunque nos ha complicado la consistencia de de los datos, nos ha sido realmente útil, ya que desde el hilo principal podemos tener acceso a la información de en qué socket esta conectado cada usuario, a parte del nick.

\end{document}
