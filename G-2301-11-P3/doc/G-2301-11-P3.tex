\documentclass{article}

\renewcommand*\familydefault{\sfdefault}

\usepackage[T1]{fontenc}
\usepackage[utf8x]{inputenc}
\usepackage[left=3cm,right=2cm,top=3cm,bottom=2cm]{geometry} % Márgenes
\usepackage{imakeidx} % Creación de índices
\usepackage{titling} % No tengo claro para qué es esto
\usepackage{fancyhdr} % Cabeceras de página
\usepackage{lastpage} % Módulo para añadir una referencia a la última página
\usepackage[spanish,es-noquoting,es-noshorthands]{babel} % Cadenas de LaTeX traducidas al español 
\usepackage{amsthm} % Paquete de matemáticas
\usepackage{framed} % Marcos 
\usepackage{mdframed} % Más marcos
\usepackage{exmath} % Nuestro querido paquete de utilidades
\usepackage{hyperref} % Para que salgan enlaces en la tabla de contenidos y el glosario
\usepackage{wrapfig} % Figuras flotantes
\usepackage{MathUnicode} % Paquete para poder poner caracteres griegos y demás cosas raras.
\usepackage{listings} % Para poner código.
\usepackage{tikz}
\usepackage{pgf-umlsd}
\usepgflibrary{arrows} % for pgf-umlsd

\fancyhf{}
\PrerenderUnicode{ÁáÉéÍíÓóÚúÑñ} % Para que salgan las tildes y demás mierdas en el título.
\fancypagestyle{plain}{%
\lhead{} 
\rhead{} 
\cfoot{\thepage\ de \pageref{LastPage}}
}

% Números en las subsecciones
\setcounter{secnumdepth}{3}

% Ajustes para los enlaces
\hypersetup{
	hyperindex,
    colorlinks,
    citecolor=black,
    filecolor=black,
    linkcolor=black,
    urlcolor=black
}

%%%%% Listings UTF8 %%%%%
\lstset{literate=%
{á}{{\'a}}1
{é}{{\'e}}1
{í}{{\'i}}1
{ó}{{\'o}}1
{ú}{{\'u}}1
{Á}{{\'A}}1
{É}{{\'E}}1
{Í}{{\'I}}1
{É}{{\'O}}1
{Ú}{{\'U}}1
}

\title{Práctica 3 Redes II - Grupo 2301 Pareja 11}
\author{Guillermo Julián Moreno y Víctor de Juan Sanz}
\date{\today}

\begin{document}
\maketitle

\begin{abstract}
En esta práctica se ha ampliado el servidor y cliente desarrollados en las prácticas anteriores para soportar SSL. Además, se ha implementado un sistema básico de envío de archivos en el cliente.
\end{abstract}

\section{Arquitectura}

\subsection{Integración de SSL}

La integración de SSL se ha hecho de la forma más transparente posible, usando funciones que reciben sockets tanto SSL como no SSL y actúan sobre ellos. Para ello hemos usado un diccionario en \texttt{ssltrans.c} que vincula cada socket a su estructura SSL correspondiente (si es un socket SSL). Así, las funciones pueden ver si el socket es SSL o no y actuar en consecuencia, recuperando todas las estructuras necesarias si es SSL.

De esta forma, no ha habido que cambiar prácticamente nada del código, salvo las funciones que manejaban sockets directamente, y en algún caso argumentos de funciones de apertura de sockets que debían decidir si se usaban sockets seguros o no.

La modificación del servidor pasa a usar dos puertos de escucha, soportando interacción entre usuarios con SSL y sin SSL gracias precisamente y transparencia de nuestra implementación de SSL, compatible con el código anterior del servidor.

\subsection{Transferencia de archivos}

\begin{figure}
  \centering

  \begin{sequencediagram}
    \newinst{ua}{Usuario A}{}
    \newinst{a}{Cliente IRC A}{}
    \newinst{b}{Cliente IRC B}{}
    \newinst{ub}{Usuario B}{}

	\mess{ua}{/fsend B file}{a}
	\mess{a}{\$FSEND file}{b}
	\mess{b}{¿Aceptar}{ub}
	\mess{ub}{/faccept}{b}
	\mess{b}{\$FACCEPT IP Puerto}{a}
	\mess{a}{archivo}{b}
	\mess{b}{Finalizado.}{ub}
	\mess{a}{Finalizado.}{ua}

  \end{sequencediagram}

  \caption{Secuencia de transferencia de archivos.}
  \label{figSeq}
\end{figure}

La transferencia de archivos se ha hecho de forma básica, tal y como se ve en la figura \ref{figSeq}. Los usuarios tienen la posibilidad de cancelar la transferencia en cualquier momento con \texttt{/fcancel}, tanto si sólo se ha hecho la petición como si se está transfiriendo el archivo.

La transferencia se hace de forma simple: primero, se envía un mensaje con el tamaño del archivo, y el receptor leerá tantos bytes como se le hayan comunicado en ese mensaje. Si no recibe todos, se producirá un error y se avisará al usuario. 

Para comunicarse con la interfaz usamos un \textit{callback}, una función que pasamos como argumento y que se llama cada vez que la transferencia cambia de estado.

\subsection{Estructuras de datos}

No se han usado nuevas estructuras de datos significativas que no se hubieran usado en anteriores prácticas. Todas están en el archivo \texttt{types.h} y correctamente documentadas en el manual (carpetas \texttt{man}, \texttt{html} y \texttt{latex/refman.pdf}).

\subsection{Código}

Pasamos ahora a describir la estructura de nuestro código y los archivos más destacados

\begin{itemize}
\item \texttt{includes} Contiene todos los archivos \texttt{.h}.
\begin{itemize}
\item \texttt{types.h} Archivo con todos los tipos y estructuras fundamentales del servidor.
\item \texttt{errors.h} Códigos de error para la interpretación del valor de retorno de nuestras funciones.
\end{itemize}
\item \texttt{src} Contiene los archivos de código principales, incluyendo \texttt{main} y las funciones específicas de IRC.
\begin{itemize}
\item \texttt{irc\_funs\_server.c} Implementaciones de cada uno de los comandos de IRC.
\item \texttt{irc\_core.c} Funciones de manejo de las estructuras de datos del servidor IRC.
\item \texttt{irc\_processor.c} Funciones auxiliares de manejo y procesado de mensajes IRC.
\end{itemize}
\item \texttt{srclib} Contiene las distintas librerías que usamos en el programa.

\begin{itemize}
\item \texttt{libcollections} Contiene \texttt{list} y \texttt{dictionary}, las dos colecciones que usamos en la práctica.
\item \texttt{libcommander} Contiene un intérprete de comandos general junto con un hilo de procesado.
\item \texttt{libjsmn} Una \href{http://zserge.bitbucket.org/jsmn.html}{librería de terceros} que incluye un analizador simple de JSON en pocas líneas de código y sin dependencias extra.
\item \texttt{libsockets} Todas las funciones necesarias para el manejo de socket, incluyendo los hilos de escucha, envío y recepción y funciones relacionadas con SSL.
\item \texttt{libstrings} Funciones auxiliares relativas al manejo de cadenas.
\item \texttt{libsysutils} Utilidades del sistema.
\item \texttt{libircgui} Funciones de la interfaz.
\end{itemize}
\item \texttt{tests} Funciones de prueba.
\item \texttt{tools} Scripts y herramientas auxiliares.
\end{itemize}

\section{Pruebas}
Hemos ido realizando test de las funciones que íbamos implementando y los hemos organizado en 3 niveles. 

\begin{itemize}
\item[1] \textbf{Test} Genera el ejecutable llamando a las \textit{suites}.
\item[2] \textbf{Suites} Colecciones de test para cada uno de los módulos, como los hilos o las funciones de IRC.
\item[3] Cada una de las \textit{suites} tiene las funciones que van probando las funcionalidades del programa.
\end{itemize}

Cada una de las funciones de test tiene una nomenclatura común que hace más sencilla la identificación de fallos: \texttt{t\_[nombre de la función a probar]\_\_[situación que se prueba]\_\_[resultado esperado]}.

Hemos realizado tests automatizados tanto de la transmisión FTP como de la comunicación SSL, de tal forma que siempre estamos seguros que funcionan correctamente.

\section{Uso}

El \textit{daemon} se arranca ejecutando el comando \texttt{bin/G-2301-11-P3-main}. Los logs se guardan en en el registro del sistema con el prefijo \textit{redirc}. En caso de querer parar el servidor, el comando \texttt{bin/G-2301-11-P3-main stop} cerrará el servidor ordenadamente.

El servidor lee parámetros de configuración del fichero \texttt{redirc.conf}. De momento los parámetros son únicamente la lista de operadores del servidor y sus contraseñas, en formato JSON no estricto.

El chat se ejecuta con \texttt{bin/G-2301-11-P3-chat}, sin parámetros. En la interfaz se puede elegir entre conexión segura o no segura.

Para ejecutar los tests, simplemente hay que ejecutar \texttt{bin/G-2301-11-P3-test}, y los resultados saldrán por pantalla. El ejecutable de pruebas tiene comandos adicionales para ejecutar sólo suites específicas (\texttt{include suite1 suite2 ...}) o excluirlas (\texttt{exclude suite1 suite2 ...}).

El cliente y el servidor de \textit{echo} se ejecutan con \texttt{bin/G-2301-11-P3-echoclient} y \texttt{bin/G-2301-11-P3-echoserver} respectivamente. Ejecutándolos con el parámetro \texttt{--help} se pueden ver las opciones para modificar los parámetros por defecto, como puerto de conexión o certificados.

\subsection{Makefile}

Nuestro Makefile cuenta con varios objetivos que pasamos a comentar:

\begin{description}
\item[no-daemon] Compilación con la bandera \texttt{NODAEMON} definida que evita que el servidor se \textit{daemonice}. Especialmente útil a la hora de depurar con \textit{gdb/lldb} o ejecutar con \textit{valgrind}.
\item[chat] Compilación del chat
\item[prsound] Compilación del programa de prueba de sonido.
\item[echoclient] Cliente echo.
\item[echoserver] Servidor echo.
\item[clean] Limpieza de ficheros ejecutables, de compilación y documentación autogenerada.
\item[test] Compilación con banderas de depuración del ejecutable de pruebas.
\item[docs] Generación de manuales con \textit{doxygen} y compilación de documentos LaTeX (esta memoria y el manual de referencia de Doxygen). 
\item[docclean] Limpieza de la documentación generada automáticamente.
\item[pack] Renombrado de archivos para cumplir con las especificaciones del enunciado y empaquetado de la práctica.
\item[certificados] Creación y verificación de los certificados de SSL.
\end{description}

\section{Conclusiones}

Durante el desarrollo de esta práctica hemos apreciado el trabajo que hicimos en las dos anteriores, centrándonos en la modularidad, robustez y uso de tests. Especialmente el último aspecto nos ha permitido hacer la migración a SSL con la seguridad de que todo seguía funcionando como antes, sin que hayamos introducido ninguna regresión.

\end{document}
